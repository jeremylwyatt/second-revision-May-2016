\section{Introduction}\label{sec:Introduction}

Prediction is central to intelligent behaviour. An agent must predict its own action effects, so as to be able to plan to achieve its goals \citep{craik1967nature}. This paper concerns prediction of the motions of a rigid object, when manipulated by a robot. The paper focuses on kinematic models of object behaviour. Within this scope, the paper makes the following contributions. First, it presents a {\em modular machine learning} solution to object motion prediction problems, and shows that, when tuned for specific objects (or contexts), this approach outperforms physics simulation.  Second, it shows transfer of learned motion models to novel objects and actions, with the first results shown for real objects. This ubiquity is precisely what rigid body simulators are designed to achieve. In this paper, we show that, if the right representations are used, transfer learning can even approach the prediction performance of a rigid body simulator on novel objects and actions.

This paper thus extends our previous work \citep{kopicki_prediction_2010,kopicki-etal-icra11},  where the core prediction algorithm was presented, and tested mostly in simulation.  This paper tests three specific hypotheses, all evaluated with respect to real objects. Hypothesis~1 (H1) is that a {\em modular learning} approach can outperform physics engines for prediction of rigid body motion.  Hypothesis~2 (H2) is that by factorising these modular predictors they can be transferred to make predictions about novel actions. Hypothesis~3 (H3) is that by factorising, learning can be transferred  to make predictions about novel shapes. We suppose that learning transfer is only effective given a good representation.

\def\stackalignment{l}
\begin{figure*}[t!]
\centerline{\stackinset{l}{0.47in}{t}{0.16in}{(a)}{\stackinset{l}{2.0in}{t}{0.16in}{(b)}{\stackinset{l}{4.3in}{t}{0.16in}{(c)}}}\includegraphics[width=0.85\textwidth]{three-prediction-problems}}
\caption{Three types of prediction problem. A robot finger is shown in blue, objects in black, and motions of the finger as dashed lines with arrows. Top row: training actions. Bottom row: an example test action. Each column represents a different problem. Sub-figure (a): Problem 1 - Action Interpolation. Subfigure (b): Problem 2 - Transfer to novel actions. Sub-figure (c): Problem 3 - Transfer to novel shapes. \label{fig:three-prediction-problems}}
\end{figure*}

The paper is structured as follows.  Section~\ref{sec:motivation} gives a motivation and background. Section~\ref{sec:schema} describes the problems to be solved, and the modular learning approach. Next, Section~\ref{sec:Representations} introduces representations of object motion, enabling a formal problem statement in Section~\ref{sec:PredictionProblem} and formulation in both regression and density estimation frameworks. Section~\ref{sec:InfoForPrediction} incorporates contact information, and Section~\ref{sec:Factors} describes how we can factor the learner by the contacts to achieve transfer learning. Section~\ref{sec:Implementation} gives implementation details, Section~\ref{sec:Experiment} the experimental method, and Section~\ref{sec:Results} the corresponding results. Section~\ref{sec:Background} reviews related work.  We finish with a discussion in Section~\ref{sec:Discussion}.