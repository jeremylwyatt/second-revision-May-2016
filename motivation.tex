
\section{The Importance of Prediction for Manipulation}
\label{sec:motivation}
The human motor system uses predictive (or forward) models of the effects that motor actions have on sensory state \citep{flanagan03,flanagan06,mehta02,witney00,johansson92}. The predictions are used for a variety of purposes, including feedforward control, coordination of motor systems, action planning, and monitoring of action plan execution. Neuroscientists have highlighted their importance for dexterous manipulation.

%\begin{quotation} the remarkable manipulative skill of the human hand is not the result of rapid sensorimotor processes, nor of fast or powerful effector mechanisms. Rather the secret lies in the way manual tasks are organised and controlled by the nervous system. At the heart of this organization is prediction. Successful manipulation requires the ability both to predict the motor commands required to grasp, lift and move objects, and to predict the sensory events that arise as a consequence of these commands. --- \citep{flanagan06}
%\end{quotation}

%There is also evidence that predicting contact events is important \citep{flanagan06}. This is because contact events during manipulation are used to assess intrinsic parameters of the manipulated object (friction, weight) as well as to monitor progress during the manipulative task. Each contact event also indicates the gain (loss) of a constraint on object motion. The motion behaviour of the object changes non-smoothly with such contacts. This means that, to manipulate skillfully, humans are likely to recruit and de-recruit forward models of object behaviour as contacts change.  Finally there is evidence that prediction is learned before control \citep{flanagan03}, suggesting that it is necessary to learn to predict object behaviour before learning to control it.

Predictive models are also useful in robot manipulation. One approach is to build a model informed by theories of mechanics \citep{mason_manipulator_1982,lynch_mechanics_1992,lynchmason96,peshkin_motion_1988,cappelleri_designing_2006,mason_mechanics_2001,flickinger2015}, to make predictions of robot and object motion under contact. Various analytic models exist, some making the quasi-static assumption, and others modelling dynamics.  To be useful for manipulation planning, their predictions must be made over a variety of timescales. To make metrically precise predictions, these models require explicit representation of intrinsic parameters, such as friction, mass, mass distribution, and coefficients of restitution. These are not trivial to estimate. Even then, model approximations can cause inaccurate predictions. Despite these challenges, analytic approaches have promise, and much work on push planning uses either purely kinematic models \citep{stillman08ijrr}, quasi static models \citep{Dogar_2010,lynchmason96} or rigid body dynamics engines \cite{zitoetal-iros12,Cosgun2011}. 

A second approach is to learn a forward model. Most such models are of qualitative effects \citep{montesano08,moldovan12,hermans11,fitzpatrick_learning_2003,ridge2010self,kroemer2014}, although metrically precise models have been learned \citep{mericli2014, scholz2010combining}. These learn action-effect correlations. We extend this to learning full rigid body motions in SE(3), in which objects may twist, slide, topple, and make and break contacts with both the robot and the environment. To achieve this we propose a modular approach. 
%. The principle has been understood for a long time \citep{JordanJacobs90, JordanRumelhart92}. It has been used to learn: the observation effects, and the non-linear dynamics, of manipulators moving in free space \citep{Ting06,Boots14,dearden2005learning}; the qualitative motions of pushed objects; and which variables influence the motion type \citep{montesano08,moldovan12,hermans11,fitzpatrick_learning_2003,ridge2010self,kroemer2014}. Finally there have been learned models of the metric motions of stable objects on a plane \citep{mericli2014}, of the motions of objects manipulated by tools \citep{Stoytchev_affordances_2008}, and of stable pushing locations \citep{hermans13}. These learning approaches all essentially learn the correlations of actions and observed effects. We extend this paradigm to learning the full rigid body motions of objects in 3D space, in which the pushed object may twist, slide, topple, and make and break contacts with both the robot and the environment. To achieve this we propose a modular architecture which we describe below.
