\section{Related Work}\label{sec:Background}

Related work is split into four broad areas: neuroscience, analytic approaches, qualitative physics, and machine learning. Prediction of motor effects on the body has long been studied in neuroscience \citep{Miall1996,flanagan03}.  MOSAIC was an early computational model of prediction and control in the cerebellum using a modular scheme \citep{Haruno_MOSAIC_2008}, where predictions can be made by convex combinations of learned predictors. Other bio-inspired modular prediction schemes were independently derived by roboticists \citep{demiris2006hierarchical}. These models all differ from ours, in that our work is the first attempt at learning to model the motions of objects with kinematic constraints. There is also evidence that infants can learn object specific motions \citep{Bahrick1995}. It is also clear that while some object knowledge may be innate \citep{spelke1994early}, object specific predictions must be learned, and are critical to our manipulation skills \citep{flanagan06}. So, in general terms, modular learning of predictions of object behaviour is cognitively plausible.

There is substantial work in robotics on analytic mechanics models of pushing \citep{mason_manipulator_1982,lynch_mechanics_1992,peshkin_motion_1988,cappelleri_designing_2006}, including both kinematic and dynamic models of manipulation effects \citep{mason_mechanics_2001}. Such analytic models are good metric predictors if their key parameters (e.g. friction) are precisely known, although qualitative predictions are robust to parameter uncertainty. They can also inform push planning under pose uncertainty \citep{brost1985planning}. These approaches are appealing in that proofs concerning the qualitative object motion can be obtained, particularly under quasi-static conditions \citep{mason1985robot,peshkin_motion_1988}. This led to methods for push planning that have some guarantees, such as completeness and optimality \citep{lynchmason96}. There is a separate body of work on qualitative models of action effects on objects, rooted in naive physics \citep{hayes1995second} and qualitative physics \citep{kuipers1986qualitative}. In a similar spirit, there is work on using physics engines to learn qualitative action effects \citep{Mugan-tamd-12}, and on high level planning of manipulation \citep{stillman08ijrr,roy2004mental} using qualitative action models. Some early ideas on push planning have reappeared in recent robots that plan pushes to enable grasps in clutter \citep{Dogar_2010}.

Learning for forward modelling has been long understood \citep{JordanJacobs90, JordanRumelhart92}. In robotics it has been used to model contactless motion, e.g. predicting the motion of an object, robot arm, or gripper in free space \citep{Ting06,Boots14,dearden2005learning}. It has also been used to learn the dynamics of an object with a single, constant contact (such as pole balancing) \citep{Schaal97,SchaalAtkeson97}. Finally, there has been work on affordance learning, and work on identifying which variables are relevant to predicting object motion \citep{montesano08,moldovan12,hermans11,fitzpatrick_learning_2003,ridge2010self,kroemer2014}. The restriction of each of these papers is that they make qualitative predictions of object motion, such as a classification of the type of motion outcome. There has also been work on predicting stable push locations \citep{hermans13}. Finally, there has been recent work in which metric motion models are learned from experience. Stoytchev \citep{Stoytchev_affordances_2008} enabled a robot to learn action effects of sticks and hook-like tools by pushing objects. This work simplifies the domain by using circular pucks as objects, and four planar motions as actions. Action outcomes were learned for various tools in a modular fashion, but without transfer learning.  In both \citep{mericli2014} and \cite{scholz2010combining} the metric planar motion of pushed objects on the plane is learned, and the learning is modularized by object, as here. In each case a small number of discrete pushing actions is tried, and motion models are planar, rather than full rigid body transformations.

Our work sits at the intersection of some these approaches. We embrace machine learning and modularity to achieve scalability, but we also explicitly model each contact. Our machine learning approach is used to make metrically precise predictions, under contact, including changing contact with the environment. In this way we try to re-achieve in a machine learning framework what only the analytic approach has attempted to date: metric predictions of motion that are transferable to novel actions and objects. 